\documentclass[12pt,a4paper]{article}
\usepackage[utf8]{inputenc}
\usepackage[brazil]{babel}
\usepackage[T1]{fontenc}
\usepackage{geometry}
\usepackage{graphicx}
\usepackage{hyperref}
\usepackage{listings}
\usepackage{xcolor}
\usepackage{amsmath}
\usepackage{amssymb}

\geometry{a4paper, left=3cm, right=2cm, top=3cm, bottom=2cm}

\definecolor{codegreen}{rgb}{0,0.6,0}
\definecolor{codegray}{rgb}{0.5,0.5,0.5}
\definecolor{codepurple}{rgb}{0.58,0,0.82}
\definecolor{backcolour}{rgb}{0.95,0.95,0.92}

\lstdefinestyle{mystyle}{
    backgroundcolor=\color{backcolour},
    commentstyle=\color{codegreen},
    keywordstyle=\color{magenta},
    numberstyle=\tiny\color{codegray},
    stringstyle=\color{codepurple},
    basicstyle=\ttfamily\footnotesize,
    breakatwhitespace=false,
    breaklines=true,
    captionpos=b,
    keepspaces=true,
    numbers=left,
    numbersep=5pt,
    showspaces=false,
    showstringspaces=false,
    showtabs=false,
    tabsize=2
}

\lstset{style=mystyle}

\title{Relatório de Projeto: Compilador SL}
\author{
    Bruno Alves Braga \\
    Matrícula: 22.1.4029 \\
    BCC328 - Construção de Compiladores I - DECOM/UFOP
}
\date{\today}

\begin{document}

\maketitle

\begin{abstract}
\end{abstract}

\tableofcontents

\section{Introdução}

\section{Metodologia}

\subsection{Estrutura sintática de SL}

\subsection{Sistema de tipos para SL}

\subsection{Inferência de tipos para SL}

\subsection{Semântica operacional para SL}


\section{Arquitetura do Compilador}

\subsection{Análise léxica}

\subsection{Análise sintática}

\subsection{Árvore de sintaxe abstrata}

\subsection{Análise semântica}

\subsection{Geração de código}

\section{Resultados e Discussão}

\subsection{Instruções de Uso}

Para utilizar o compilador de Sl, inicie os containers com os comandos:

docker-compose up -d
docker-compose exec sl bash

Dentro do diretório workspace/tp1/src/Sl/Lexer execute o comando: alex Lexer.x par worka gerar o analisador léxico.

Para executar o projeto, dentro do diretório workspace, utilize o comando de auda para ver as opções:
cabal run sl -- --help 

Caso queira utilziar o analisador léxico:
cabal run sl -- --lexer <arquivo teste>


\subsection{Testes Realizados}


\subsection{Limitações}

\section{Conclusão}

\section{Referências}

\begin{thebibliography}{9}
\bibitem{hutton2016}
Hutton, G. (2016). \emph{Programming in Haskell}. Cambridge University Press.

\bibitem{appel1998}
Appel, A. W. (1998). \emph{Modern Compiler Implementation in ML}. Cambridge University Press.

\bibitem{webassembly2023}
WebAssembly Community Group. (2023). \emph{WebAssembly Specification}.
\end{thebibliography}

\end{document}
